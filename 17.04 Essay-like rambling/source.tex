\documentclass[12pt]{article}
\title{\textsc{A Few Thoughts on "Against an Increasingly User-Hostile Web"}}
\date{\vspace{-5ex}}
\author{Vasiliy Vekovshinin}
\pagenumbering{gobble}

\begin{document}
\maketitle
The Internet has become an integral part of our lives and is objectively a prosperous invention. It is used for work, studies, communication and entertainment by all sorts of people. A layperson would be hard-pressed to find any meaningful flaws of this interconnected miracle. However, arguably there are.  Recently I had a pleasure of reading an article titled "Against an Increasingly User-Hostile Web". While at first I expected it to be a boring "grass was greener back in the day" whine, it turned out to be reasonable and, in some sense, disturbing. I'd like to discuss some points of the article and share my opinion on it.
\newline

First of all, the author analyzes the web's history: it was mostly used by scientists until 1993, when CERN released it to public domain. Soon after the web started to get bloated and intrusive, which is exemplified by the case of Cambridge Analytica. Personally, I missed the outrage when it happened but, looking back, this was an alarming precedent that showed the web's harmful potential. Indeed, using advertisement to alter people's political opinion sure sounds like George Orwell material! I didn't use to pay much attention to people who demonized companies selling users' data for advertisement, now I understand these people much better. I realized that big data companies are no-joke, they could (and probably do) lead us into some dystopian future. 
\newline

The second (third?) point is about trackers consuming traffic and issuing third-party requests. The author conducts an experiment: how much of transferred data is actually related to the contents of a web page. Well, I'm not quite impressed with this one - most users already know that god-damned ads and trackers eat a large chunk of precious ethernet/wi-fi molecules. It's safe to say that this is but an inconvenience, and doesn't compare to the actual threat of trackers, described in the previous subtopic.
\newline

Moving on, the article warns us of private companies taking over. Undeniably, these corps own many popular services and can dictate the rules: you're either with them or out. This creates another ground for possible  technocratic hell: a social network doesn't want you speaking ill of a politician who is friends with its founder, does it? The author advocates for using open-source analogs, as well as not keeping eggs in one basket and keeping in mind any service might shut down as soon as tomorrow. This is what I 100\% agree with. In fact, due to this article (and by checking out what my Python teacher uses) I started to use Brave as my main browser, as I don't support Google's endless (and oftentimes, failing) expansion.
\newline

Finally, the author presents some guidelines for users and 'ethic' web devs. As helpful as they are, we should keep in mind that it's not a complete list. Most users are still ignorant about their privacy, and due to them predatory companies will likely thrive for some time, probably creating even more twisted ways of keeping users under the thumb. So we need to continuously adapt, boycott harmful practices and keep our "digital footprint" as small as possible. Otherwise, not only our digital, but our very physical future will get compromised and the next calendar we'll buy will be for the year 1984.






\end{document}